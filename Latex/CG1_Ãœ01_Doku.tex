\documentclass[a4paper,10pt]{scrreprt}

\usepackage[utf8]{inputenc}
\usepackage{ngerman}


\title{Bericht Todesstern U1}
\author{Charline Waldrich, Robert Ullmann, Julian Dobrot}
\date{30. Oktober 2015}

\begin{document}

\maketitle
\tableofcontents

\section{Aufgabenstelung}
\subsection{ImageViewer}
Der ImageViewer ist ein Programm mit dem eine Bilddatei von der Festplatte ausgewählt werden kann und dieses in der GUI angezeigt wird. 
\subsection{ImageSaver}
Der ImageSaver ist ein Programm walches ein Bild erzeugt, dass so groß ist wie das Fenster.
Das Bild ist Schwarz und beinhaltet einen diagonalen roten Strich. Über eine Menüzeile kann das Bild in den Formaten PNG und JPG gespeichert werden.
\subsection{Matrizen- und Vectorenbibliothek}
Die Matritzen- und Vektorenbibliothek beinhaltet die folgenden vier Klassen 

\section{Lösungsstrategien}
\section{Implementierungen}
\section{Besondere Probleme oder Schwierigkeiten bei der Bearbeitung}
\section{Zeitbedarf}

\end{document}