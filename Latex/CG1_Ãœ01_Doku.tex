\documentclass[a4paper,10pt]{ltxdoc}

\usepackage[utf8]{inputenc}
\usepackage{ngerman}

\usepackage{array}
\usepackage{amsmath}

\title{Bericht Todesstern U1}
\author{Charline Waldrich, Robert Ullmann, Julian Dobrot}
\date{30. Oktober 2015}

\begin{document}

\maketitle
\tableofcontents
\pagebreak

\section{Aufgabenstelung}

\subsection{ImageViewer}
Der ImageViewer ist ein Programm mit dem eine Bilddatei von der Festplatte ausgewählt werden kann und dieses in der GUI angezeigt wird. 
\subsection{ImageSaver}
Der ImageSaver ist ein Programm walches ein Bild erzeugt, dass so groß ist wie das Fenster.
Das Bild ist Schwarz und beinhaltet einen diagonalen roten Strich. Über eine Menüzeile kann das Bild in den Formaten PNG und JPG gespeichert werden.
\subsection{Matrizen- und Vectorenbibliothek}
Die Matritzen- und Vektorenbibliothek beinhaltet die folgenden vier Klassen 

\section{Lösungsstrategien}
\subsection{ImageViewer}
Der ImageViewer öffnet beim starten einen Dialog um eine Bilddatei im PNG oder JPG- Format auszuwählen. Das ausgewählte Bild wird in einem Fenster, passend zur Größe, angezeigt.
\subsection{ImageSaver}

\subsection{Matrizen- und Vectorenbibliothek}

\section{Implementierungen}
\subsection{ImageViewer}
Der ImageViewer besteht aus zwei Methoden. Die erste Methode ist die Java-FX \textbf{start}-Methode, welche das Layout beinhaltet. Dieses besteht aus einer BorderPane und darin enthalten eine ImageView, welche das Bild anzeigt. \\
Die zweite Methode ist die \textbf{openFileDialog}-Methode. Diese Methode dient der Auswahl des Bildes und gibt das ausgewählte File zurück an die start-Methode.
\subsection{ImageSaver}

\subsection{Matrizen- und Vectorenbibliothek}

\section{Besondere Probleme oder Schwierigkeiten bei der Bearbeitung}
Bei der Implementierung des ImageViewers kam es zu keinen besondern Problemen oder Schwierigikeiten.
\section{Zeitbedarf}
\begin{center}
\begin{tabular}{cr}
ImageViewer	  \	&80 min	\\
ImageSaver 	\	&120 min	\\
Bibliotheken \	&120 min	\\
Bericht  \		&180 min	 \\
	\hline
	&500 min
\end{tabular}
\end{center}


\end{document}