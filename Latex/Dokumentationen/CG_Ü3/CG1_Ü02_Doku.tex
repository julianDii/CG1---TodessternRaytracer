\documentclass[14pt]{extarticle}
\usepackage[utf8]{inputenc}
\usepackage{ngerman}
\usepackage{array}
\usepackage{amsmath}
\usepackage{graphicx}
\title{Bericht Todesstern U3}
\author{Charline Waldrich, Robert Ullmann, Julian Dobrot}
\date{13. november 2015}

\begin{document}

\maketitle
\pagebreak
\tableofcontents


\section{Aufgabenstellung}
Implementierung von Beleuchtung im todesstern Raytracer.
\subsection{Änderungen an bestehenden Klassen}
Die Klasse World wurde um den Typ Color für das Ambiente Licht erweitert. Alle Objekte und Lichtquellen werden weiterhin in einer Liste an die Welt übergeben.
\subsubsection{Änderungen am Hit-Objekt}
Das Hit-Objekt besitzt nun die Normale vom Typ Normal3 mit den Namen ``nor''.
\subsubsection{Änderungen an der Geometrie-Klasse und am Dreieck}
Die Klasse Geometry bekommt nun das Material übergeben, welches in der Materialklasse die Methode colorFor aufruft. 

\subsection{Licht}
Die Klasse Licht, erzeugt die Farbe und die Beleuchtung. Dabei wird der Vektor berechnet, welcher zwischen Licht und Objekt entsteht.
\subsubsection{PointLight}
PointLight erwartet die Position und die Farbe der Lichtquelle und berechnet den Vektor zwischen Licht und Objekt. Dabei wird die Scene beleuchtet, die gleichmäßig in alle Richtungen strahlt.
\subsubsection{DirectionalLight}
Die Direktionale Lichtquelle ist ähnlich der Punktlichtquelle, nur wird dabei die Scene überall gleichmäßig beleuchtet.
\subsubsection{SpotLight}
Die Spotlichquelle strahlt aus einem übergebenen Punkt in eine übergebene Richtung innerhalb eines festgelegten Winkels.

\subsection{Material}
Die Klasse Material ist die Basisklasse aller Materialien-
\subsubsection{SingleColorMaterial}
Das SingleColorMaterial, beachtet keine Lichtquellen und gibt einfach nur die Farbe des Objektes zurück.
\subsubsection{LambertMaterial}
Das LambertMaterial ist ein reflektierendes Material ohne Glanzpunkt und die Reflektion ist die Farbe des ambienten Lichtes.
\subsubsection{PhongMaterial}
Das Phong-Material ist ein reflektierendes Material mit Glanzpunkt und die Reflektion ist die übergebene Farbe zusätzlich zu der übergebenen Farbe der Reflektion.
\section{Implementierungen}

\section{Zeitbedarf}
\begin{center}
\begin{tabular}{cr}
Änderungen an bestehenden Klassen \	&60 min	\\
Licht	  \	&240 min	\\
Material 	\	&180 min	\\
Welt \	&60 min	\\
Demo \	&240 min	\\
Bericht  \		&180 min	 \\
	\hline
	&960 min
\end{tabular}
\end{center}

\section{Quellen}

\end{document}
